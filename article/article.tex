\documentclass[]{article}

\usepackage{fullpage}
%\usepackage{amsmath}
\usepackage{natbib}
\usepackage[russian]{babel}
\usepackage[utf8]{inputenc}

%opening
\title{Advertisement detection in Russian Wikipedia}
\author{versusvoid}

\begin{document}

\maketitle

\begin{abstract}

TODO

\end{abstract}

\section{Introduction}

Wikipedia is well known for it's attempt to be a neutral source of knowledge. 
Neutrality is achieved by collaborative editing with contributions from users 
holding different opinions. Yet there are many articles (or sections in big 
articles) edited by couple of users at most and not receiving much attention 
form others. Sometimes, whether by intent or by affect, these give impression 
of advertisement - silencing negative aspects and exaggerating positive. 
Automatic detection and correction suggestions will speedup improvement of
articles quality.

\section{Related works}

There are no known results for Russian Wikipedia, but this field is well 
explored both in English-only and general language setting.


In \citealp{bhosale:emnlp13} LogitBoost classifier was used on obvious 
stylometric along with usual n-gram and PCFG features, giving F1 score of 0.93.

In \citealp{deep-learning-vectors} authors explored usage of vector models and
achieved 0.96 F1 score, but used synthetic data as input.

\section*{References}


\bibliographystyle{authordate1}
\bibliography{bibliography}

\end{document}
